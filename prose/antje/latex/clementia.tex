\section{De clementia}
scribere de clementia, nero caesar, institui, ut quodam modo speculi vice fungerer et te tibi ostenderem perventurum ad voluptatem maximam omnium. quamvis enim recte factorum verus fructus sit fecisse nec ullum virtutum pretium \textbf{dignum illis extra ipsas sit, iuvat inspicere et} circumire bonam conscientiam, tum immittere oculos in hanc immensam multitudinem discordem, seditiosam, impotentem, in perniciem alienam suamque pariter exsultaturam, si hoc iugum fregerit, et ita loqui secum: 'egone ex omnibus mortalibus placui electusque sum, qui in terris deorum vice fungerer? ego vitae necisque gentibus arbiter; qualem quisque sortem statumque habeat, in mea manu positum est; quid cuique mortalium fortuna datum velit, meo ore pronuntiat; ex nostro responso laetitiae causas populi urbesque \textbf{concipiunt; nulla pars usquam nisi volente} propitioque me floret; haec tot milia gladiorum, quae pax mea comprimit, ad nutum meum stringentur; quas nationes funditus excidi, quas transportari, quibus libertatem dari, quibus eripi, quos \textbf{reges mancipia fieri quorumque capiti} regium circumdari decus oporteat, quae ruant urbes, quae oriantur, mea iuris dictio est. in hac tanta facultate rerum non ira me ad iniqua supplicia compulit, non iuvenilis impetus, non temeritas hominum et contumacia, quae saepe tranquillissimis quoque pectoribus patientiam extorsit, non ipsa ostentandae per terrores potentiae dira, sed frequens magnis imperiis \textbf{gloria. conditum, immo constrictum apud me} ferrum est, summa parsimonia etiam vilissimi sanguinis; nemo non, cui alia desunt, hominis nomine apud me gratiosus est. severitatem abditam, at clementiam in procinctu habeo; sic me custodio, tamquam legibus, quas ex situ ac tenebris in lucem evocavi, rationem redditurus sim. alterius aetate prima motus sum, alterius ultima; alium dignitati donavi, alium humilitati; quotiens nullam inveneram misericordiae causam, mihi peperci. hodie dis immortalibus, si a me rationem repetant, adnumerare genus humanum paratus sum.' potes hoc, \textbf{caesar, audacter praedicare: omnia, quae in} fidem tutelamque tuam venerunt, tuta haberi, nihil per te neque vi neque clam adimi rei publicae. rarissimam laudem et nulli adhuc principum concessam concupisti innocentiam. non perdit operam nec bonitas ista tua singularis ingratos aut malignos aestimatores nancta est. refertur \textbf{tibi gratia; nemo unus homo uni homini tam} carus umquam fuit, quam tu populo romano, magnum longumque eius bonum. sed ingens tibi onus imposuisti; nemo iam divum augustum nec ti. caesaris \textbf{prima tempora loquitur nec, quod te imitari} velit, exemplar extra te quaerit; principatus tuus ad gustum exigitur. difficile hoc \textbf{fuisset, si non naturalis tibi ista} bonitas esset, sed ad tempus sumpta. nemo enim potest personam diu ferre, ficta cito in naturam suam recidunt; quibus veritas subest quaeque, ut ita dicam, ex solido enascuntur, tempore ipso in maius meliusque procedunt. magnam adibat aleam populus romanus, cum incertum esset, quo se ista tua nobilis indoles daret; iam vota publica in tuto sunt; nec enim periculum est, ne te subita tui capiat oblivio. facit quidem avidos nimia felicitas, nec tam temperatae cupiditates sunt umquam, ut in eo, quod contigit, desinant; gradus a magnis ad maiora fit, et spes improbissimas conplectuntur insperata adsecuti; omnibus tamen nunc civibus tuis et haec confessio \textbf{\textbf{exprimitur esse felices et illa nihil iam} his} accedere bonis posse, nisi ut perpetua sint. multa illos cogunt ad hanc confessionem, qua nulla in homine tardior est: securitas alta, adfluens, ius supra omnem iniuriam positum; obversatur oculis laetissima forma rei publicae, cui ad summam libertatem nihil deest nisi pereundi licentia. praecipue tamen aequalis ad maximos imosque pervenit clementiae tuae admiratio; cetera enim bona pro portione fortunae suae quisque sentit aut exspectat maiora minoraque, ex clementia omnes idem sperant; nec est quisquam, cui tam valde innocentia sua placeat, ut non stare in conspectu clementiam paratam humanis erroribus gaudeat. esse autem aliquos scio, qui clementia pessimum quemque putent sustineri, quoniam nisi post crimen supervacua est et sola haec virtus inter innocentes \textbf{cessat. \textbf{sed primum omnium, sicut medicinae} apud aegros} usus, etiam apud sanos honor est, ita clementiam, quamvis poena digni invocent, etiam innocentes colunt. deinde habet haec in persona quoque innocentium locum, quia interim fortuna pro culpa est; nec innocentiae tantum clementia succurrit, sed saepe virtuti, quoniam quidem condicione temporum incidunt quaedam, \textbf{quae possint laudata \textbf{puniri. adice, quod} magna pars hominum est, quae} reverti ad innocentiam possit, si poenae remissio fuerit. non tamen volgo ignoscere decet; nam ubi discrimen inter malos bonosque sublatum est, confusio sequitur et vitiorum eruptio; itaque adhibenda moderatio est, quae sanabilia ingenia distinguere a \textbf{\textbf{deploratis sciat. nec promiscuam habere} ac} volgarem clementiam oportet nec abscisam; nam tam omnibus ignoscere crudelitas quam nulli. modum tenere debemus; sed quia difficile est temperamentum, quidquid aequo plus futurum est, in partem humaniorem praeponderet. sed haec suo melius loco dicentur. nunc in tres partes omnem hanc materiam dividam. prima erit manumissionis; secunda, quae naturam clementiae habitumque demonstret: nam cum sint vitia quaedam virtutes imitantia, non possunt secerni, nisi signa, quibus dinoscantur, impresseris; tertio loco quaeremus, quomodo ad hanc virtutem perducatur animus, quomodo confirmet eam et usu suam faciat. nullam ex omnibus virtutibus homini magis convenire, cum sit nulla humanior, constet necesse est non solum inter nos, qui hominem sociale animal communi bono genitum videri volumus, sed etiam inter illos, qui hominem voluptati donant, \textbf{quorum omnia dicta factaque ad utilitates} suas spectant; nam si quietem petit et otium, hanc virtutem naturae suae nanctus est, quae pacem amat et manus retinet. nullum tamen clementia ex omnibus magis quam regem aut principem decet. ita enim magnae vires decori gloriaeque sunt, si illis salutaris potentia est; nam pestifera vis est valere ad nocendum. illius demum magnitudo stabilis fundataque est, quem omnes tam supra se esse quam pro se sciunt, cuius curam excubare pro salute singulorum atque universorum cottidie experiuntur, \textbf{quo \textbf{procedente non, \textbf{tamquam malum aliquod aut}} noxium animal e} cubili prosilierit, diffugiunt, sed tamquam ad clarum ac beneficum sidus certatim advolant. obicere se pro illo mucronibus insidiantium paratissimi et substernere corpora sua, si per stragem illi humanam iter ad salutem struendum sit, somnum eius nocturnis excubiis muniunt, latera obiecti circumfusique defendunt, incurrentibus periculis se opponunt. non est hic sine ratione populis urbibusque consensus sic protegendi amandique reges et se suaque iactandi, quocumque desideravit imperantis salus; nec haec vilitas \textbf{sui est aut dementia pro uno capite tot} milia \textbf{excipere \textbf{ferrum ac multis mortibus unam} animam redimere} nonnumquam senis et invalidi. quemadmodum totum corpus animo deservit et, cum hoc tanto maius tantoque speciosius sit, ille in occulto maneat tenuis et in qua sede latitet incertus, tamen manus, pedes, oculi negotium illi gerunt, illum haec cutis munit, illius iussu iacemus aut inquieti discurrimus, cum ille imperavit, sive avarus dominus est, mare \textbf{lucri causa scrutamur, sive ambitiosus,} iam dudum dextram flammis obiecimus aut voluntarii terram subsiluimus, sic haec immensa multitudo unius animae circumdata illius spiritu regitur, illius ratione flectitur pressura se ac fractura viribus suis, nisi consilio sustineretur. suam itaque incolumitatem amant, cum pro uno homine denas legiones in aciem deducunt, cum in primam frontem procurrunt \textbf{et adversa volneribus pectora ferunt, ne} imperatoris sui signa vertantur. ille est enim vinculum, per quod res publica cohaeret, ille spiritus vitalis, quem haec tot milia trahunt nihil ipsa per se futura nisi onus et praeda, si mens illa imperii subtrahatur. rege incolumi mens \textbf{omnibus una; amisso rupere fidem. \textbf{hic} casus romanae pacis exitium erit,} hic tanti fortunam populi in ruinas aget; tam diu ab isto periculo aberit hic populus, quam diu sciet ferre frenos, quos si quando abruperit vel aliquo casu discussos reponi sibi passus non erit, haec unitas et hic maximi imperii contextus in partes multas dissiliet, \textbf{idemque huic urbi finis dominandi erit, qui} parendi fuerit. ideo principes regesque et quocumque alio nomine sunt tutores status publici non est mirum amari ultra privatas etiam necessitudines; nam si sanis hominibus \textbf{publica privatis potiora sunt, sequitur, ut} is quoque carior sit, in quem \textbf{se res publica convertit. olim enim ita se} induit rei publicae caesar, ut seduci alterum non posset sine utriusque pernicie; nam et illi viribus opus est et huic capite. longius videtur recessisse a proposito oratio mea, at mehercules rem ipsam premit. nam si, quod adhuc colligit, tu animus rei publicae tuae es, illa corpus tuum, vides, ut puto, quam necessaria sit clementia; tibi enim parcis, cum videris alteri parcere. parcendum itaque est etiam improbandis civibus \textbf{non aliter quam membris languentibus, et, si} quando misso sanguine opus est, sustinenda est manus, ne ultra, quam necesse sit, incidat. \textbf{est ergo, ut dicebam, clementia omnibus} quidem hominibus secundum naturam, maxime tamen decora imperatoribus, quanto plus habet apud illos, quod servet, quantoque in maiore materia apparet. quantulum enim nocet privata crudelitas! principum \textbf{saevitia bellum est. cum autem virtutibus inter} se sit concordia nec ulla altera melior aut honestior sit, quaedam tamen quibusdam personis aptior est. decet magnanimitas quemlibet mortalem, etiam illum, infra quem nihil est; quid enim maius aut fortius quam malam fortunam retundere? haec tamen magnanimitas in bona \textbf{\textbf{fortuna laxiorem locum habet meliusque} in} tribunali quam in plano conspicitur. clementia, in quamcumque domum pervenerit, eam felicem tranquillamque praestabit, sed in \textbf{regia, quo rarior, eo mirabilior. quid} enim est memorabilius quam eum, cuius irae nihil obstat, cuius graviori sententiae ipsi, qui pereunt, adsentiuntur, quem nemo interpellaturus est, immo, si vehementius excanduit, ne deprecaturus est quidem, ipsum sibi manum inicere et potestate \textbf{sua in melius placidiusque uti hoc ipsum} cogitantem: 'occidere contra legem nemo non potest, servare nemo praeter me'? magnam fortunam magnus animus decet, qui, nisi se ad illam extulit et altior stetit, illam quoque infra ad terram \textbf{deducit; magni autem animi proprium est placidum esse} tranquillumque et iniurias atque offensiones superne despicere. muliebre est furere in ira, ferarum vero nec generosarum quidem praemordere et urguere proiectos. elephanti leonesque transeunt, quae impulerunt; ignobilis bestiae pertinacia est. non decet regem saeva nec inexorabilis ira, non multum enim supra eum eminet, cui se irascendo exaequat; at si dat vitam, si dat dignitatem periclitantibus et meritis amittere, facit, quod nulli nisi rerum potenti licet; vita enim etiam superiori eripitur, numquam nisi inferiori datur. servare proprium est excellentis fortunae, quae numquam magis suspici debet, quam cum illi contigit idem posse quod dis, quorum beneficio in lucem edimur tam boni quam mali. deorum itaque sibi animum adserens princeps alios ex civibus suis, quia utiles bonique sunt, libens videat, alios in numerum relinquat; quosdam esse gaudeat, quosdam patiatur. cogitato, in hac civitate, \textbf{in qua turba per latissima itinera sine} intermissione defluens eliditur, quotiens aliquid obstitit, quod cursum eius velut torrentis rapidi moraretur, in qua tribus eodem tempore theatris caveae postulantur, in qua consumitur quicquid terris omnibus aratur, quanta solitudo ac vastitas futura sit, si nihil relinquitur, nisi quod iudex severus absolverit. quotus quisque ex \textbf{quaesitoribus \textbf{\textbf{est, qui non ex ipsa ea lege} teneatur,} qua} quaerit? quotus quisque accusator vacat culpa? et nescio, an nemo ad dandam veniam difficilior sit, quam qui illam petere saepius meruit. peccavimus omnes, alii gravia, alii leviora, alii ex destinato, alii forte impulsi aut aliena nequitia ablati; alii in bonis consiliis parum fortiter stetimus et innocentiam inviti ac retinentes perdidimus; nec deliquimus tantum, sed usque \textbf{ad \textbf{\textbf{extremum aevi delinquemus. etiam si}} quis} tam bene iam purgavit animum, ut nihil obturbare eum amplius possit ac fallere, ad innocentiam tamen peccando pervenit. quoniam deorum feci mentionem, \textbf{optime hoc exemplum principi constituam, ad quod} formetur, ut se talem esse civibus, quales sibi deos velit. expedit ergo habere inexorabilia peccatis atque erroribus numina, expedit usque ad ultimam infesta perniciem? et quis regum erit tutus, cuius non membra haruspices colligant? quod si di placabiles et aequi delicta potentium non statim fulminibus persequuntur, quanto aequius est hominem hominibus praepositum miti animo exercere imperium et cogitare, uter mundi status gratior oculis pulchriorque sit, sereno et puro die, an cum fragoribus crebris omnia quatiuntur et ignes hinc atque illinc micant! atqui non alia facies est quieti moratique imperii quam sereni caeli et nitentis. crudele regnum turbidum tenebrisque obscurum est, inter trementes et ad repentinum sonitum expavescentes ne eo quidem, \textbf{qui \textbf{omnia perturbat, inconcusso. facilius}} privatis ignoscitur pertinaciter se vindicantibus; possunt enim laedi, dolorque eorum ab iniuria venit; timent praeterea contemptum, et non rettulisse laedentibus gratiam infirmitas videtur, non clementia; at cui ultio in facili est, is omissa ea certam laudem mansuetudinis consequitur. humili loco positis exercere manum, litigare, in rixam procurrere ac morem irae suae gerere liberius est; leves inter paria ictus sunt; regi vociferatio quoque verborumque intemperantia non ex maiestate est. grave putas eripi loquendi arbitrium regibus, quod humillimi habent. \textbf{'ista' inquis 'servitus est, non imperium.' quid?} tu non experiris istud nobis esse, tibi servitutem? alia condicio est eorum, qui in turba, quam non excedunt, latent, quorum et virtutes, ut appareant, diu luctantur et vitia tenebras habent; vestra facta dictaque rumor excipit, et ideo nullis magis curandum est, qualem \textbf{famam habeant, quam \textbf{qui, qualemcumque meruerint,} magnam habituri} sunt. quam multa tibi non licent, quae nobis beneficio tuo licent! possum in qualibet parte urbis solus incedere sine timore, quamvis nullus sequatur comes, nullus sit domi, \textbf{nullus ad latus gladius; tibi in tua pace} armato vivendum est. aberrare a fortuna tua non potes; obsidet te et, quocumque descendis, magno apparatu sequitur. est haec summae magnitudinis servitus non posse fieri minorem; sed cum dis tibi communis ipsa necessitas est. nam illos quoque caelum adligatos tenet, nec magis illis descendere datum est quam tibi tutum: fastigio tuo adfixus es. nostros motus pauci sentiunt, prodire nobis ac recedere et mutare habitum sine sensu publico licet; tibi non magis quam soli latere contingit. multa circa te \textbf{lux est, omnium in istam conversi oculi sunt;} prodire te putas? oriris. loqui non potes, nisi ut vocem tuam, quae ubique sunt gentes, excipiant; irasci non potes, nisi ut omnia tremant, quia neminem adfligere, \textbf{nisi ut, quidquid circa fuerit, quatiatur.} ut fulmina paucorum periculo cadunt, omnium metu, sic animadversiones magnarum potestatum terrent latius quam nocent, non sine causa; non enim, quantum fecerit, sed quantum facturus sit, cogitatur in eo, qui omnia \textbf{potest. adice \textbf{nunc, quod privatos homines ad} accipiendas} iniurias opportuniores acceptarum patientia facit, regibus certior est ex mansuetudine securitas, quia frequens vindicta paucorum odium opprimit, omnium inritat. voluntas oportet ante saeviendi quam causa deficiat; alioqui, quemadmodum praecisae arbores plurimis ramis repullulant et multa satorum genera, ut densiora surgant, reciduntur, ita regia crudelitas auget inimicorum numerum tollendo; parentes enim liberique eorum, qui interfecti sunt, et propinqui et amici in locum singulorum succedunt. hoc quam verum sit, admonere te exemplo domestico volo. divus augustus fuit mitis princeps, si quis illum a principatu suo aestimare incipiat; in communi quidem rei publicae gladium movit. cum hoc aetatis esset, \textbf{quod tu nunc es, duodevicensimum egressus} annum, iam pugiones in sinum amicorum absconderat, iam insidiis antonii consulis latus petierat, iam fuerat collega proscriptionis. sed cum annum quadragensimum transisset et in gallia moraretur, delatum est ad eum indicium cinnam, stolidi ingenii virum, \textbf{insidias ei struere; dictum est, et ubi et} quando et quemadmodum adgredi vellet; unus ex consciis deferebat. constituit se ab eo vindicare et consilium amicorum advocari iussit. nox illi inquieta erat, cum cogitaret adulescentem nobilem, hoc detracto integrum, cn. pompei nepotem, damnandum; iam unum hominem occidere non poterat, cui antonius proscriptionis edictum inter cenam dictarat. gemens subinde voces varias \textbf{emittebat et inter se contrarias: 'quid} ergo? ego percussorem meum securum ambulare patiar me sollicito? ergo non dabit poenas, qui tot civilibus bellis frustra petitum caput, tot navalibus, tot pedestribus proeliis incolume, postquam terra marique pax parata est, non occidere constituat, sed immolare?' nam sacrificantem placuerat adoriri. rursus silentio interposito maiore multo voce sibi quam cinnae irascebatur: 'quid vivis, si perire te tam multorum interest? quis finis \textbf{\textbf{erit \textbf{suppliciorum? quis sanguinis? ego} sum}} nobilibus adulescentulis \textbf{expositum caput, in quod mucrones acuant; \textbf{non} \textbf{est tanti vita, si, ut ego non peream, tam}} multa perdenda sunt.' \textbf{interpellavit tandem illum \textbf{livia uxor} et: 'admittis' inquit 'muliebre} consilium? fac, quod medici solent, qui, ubi usitata remedia non procedunt, temptant contraria. severitate nihil adhuc profecisti; salvidienum lepidus secutus est, lepidum murena, murenam caepio, caepionem egnatius, ut alios taceam, quos tantum ausos pudet. nunc tempta, quomodo tibi cedat clementia; ignosce cinnae. deprensus est; iam nocere tibi non potest, prodesse famae tuae potest.' gavisus, sibi quod advocatum invenerat, uxori quidem gratias \textbf{egit, renuntiari autem extemplo amicis, quos} in consilium rogaverat, imperavit et cinnam unum ad se accersit dimissisque omnibus e cubiculo, cum alteram cinnae poni cathedram \textbf{iussisset: 'hoc' inquit 'primum a te peto, ne me} loquentem interpelles, ne medio sermone meo proclames; dabitur tibi loquendi liberum tempus. ego te, cinna, cum in hostium castris invenissem, non factum tantum mihi inimicum sed natum, servavi, patrimonium tibi omne concessi. hodie tam felix et tam dives es, ut victo victores invideant. sacerdotium tibi petenti praeteritis compluribus, quorum parentes mecum militaverant, dedi; cum sic de te meruerim, occidere me constituisti.' cum ad hanc vocem exclamasset procul hanc ab se abesse dementiam: 'non praestas' inquit 'fidem, cinna; convenerat, ne interloquereris. occidere, inquam, me paras'; adiecit locum, socios, diem, ordinem insidiarum, cui commissum esset ferrum. et cum defixum videret nec ex conventione iam, sed ex conscientia tacentem: 'quo' inquit 'hoc animo facis? ut ipse sis princeps? male mehercules cum populo romano agitur, si tibi ad imperandum nihil praeter me obstat. domum tueri tuam non potes, nuper libertini hominis \textbf{gratia in privato iudicio superatus} es; adeo nihil facilius potes quam contra caesarem advocare. cedo, si spes tuas solus impedio, paulusne te et fabius maximus et cossi et servilii ferent tantumque agmen nobilium non inania nomina praeferentium, sed eorum, qui imaginibus suis decori sint?' ne totam eius orationem repetendo magnam partem voluminis occupem diutius enim quam duabus horis locutum esse constat, cum hanc poenam, qua sola erat contentus futurus, \textbf{extenderet: 'vitam' inquit 'tibi, cinna, iterum do,} prius hosti, nunc insidiatori ac parricidae. ex hodierno die inter nos amicitia \textbf{incipiat; \textbf{contendamus, utrum ego meliore} \textbf{fide tibi} vitam dederim an tu debeas.' post} hoc detulit ultro consulatum questus quod non auderet petere. amicissimum fidelissimumque habuit, heres solus \textbf{illi fuit. nullis amplius insidiis ab} ullo petitus est. ignovit abavus tuus victis; nam si non ignovisset, quibus imperasset? sallustium et cocceios et deillios et totam cohortem primae admissionis ex adversariorum castris conscripsit; iam domitios, messalas, asinios, cicerones, quidquid floris erat in civitate, clementiae suae debebat. ipsum lepidum quam diu mori passus est! per multos annos tulit ornamenta principis retinentem et pontificatum maximum non nisi mortuo illo transferri in se passus est; maluit enim illum honorem vocari quam spolium. haec eum clementia ad salutem securitatemque perduxit; haec gratum ac favorabilem reddidit, quamvis nondum subactis populi romani cervicibus manum imposuisset; haec hodieque praestat illi famam, quae vix vivis principibus servit. deum esse non tamquam iussi credimus; bonum fuisse principem augustum, bene illi \textbf{parentis nomen convenisse fatemur} ob nullam aliam causam, quam quod contumelias quoque suas, quae acerbiores principibus solent esse quam iniuriae, nulla crudelitate exsequebatur, quod probrosis in se dictis adrisit, quod dare \textbf{illum poenas apparebat, cum exigeret, quod,} quoscumque ob adulterium filiae suae damnaverat, adeo non occidit, ut dimissis, quo tutiores essent, diplomata daret. hoc est ignoscere, cum scias multos futuros, qui pro te irascantur et tibi sanguine alieno gratificentur, non dare tantum salutem, sed praestare. haec augustus senex aut iam in senectutem annis vergentibus; in adulescentia caluit, arsit ira, multa fecit, ad quae invitus oculos retorquebat. comparare nemo mansuetudini tuae audebit divum augustum, etiam si in certamen iuvenilium annorum deduxerit senectutem plus quam maturam; fuerit moderatus et clemens, nempe post mare actiacum romano cruore infectum, nempe post fractas in sicilia classes et \textbf{suas \textbf{et alienas, nempe post perusinas} aras} et proscriptiones. ego vero clementiam non voco lassam crudelitatem; haec est, caesar, clementia vera, quam tu praestas, quae non saevitiae paenitentia coepit, nullam habere maculam, numquam civilem sanguinem fudisse; haec est in maxima potestate verissima animi temperantia et \textbf{humani generis comprendens ut sui amor non} cupiditate aliqua, non temeritate ingenii, non priorum principum exemplis corruptum, quantum sibi cives suos liceat, experiendo temptare, sed hebetare aciem imperii sui. praestitisti, caesar, civitatem incruentam, et hoc, quod magno animo gloriatus es nullam te toto orbe stillam cruoris humani \textbf{misisse, eo maius est mirabiliusque,} quod nulli umquam citius gladius commissus est. clementia ergo non tantum honestiores sed tutiores praestat ornamentumque imperiorum est simul et certissima \textbf{salus. quid enim \textbf{\textbf{est, cur reges consenuerint} liberisque} ac} nepotibus tradiderint regna, tyrannorum exsecrabilis ac brevis potestas sit? quid interest inter tyrannum ac regem species enim ipsa fortunae ac licentia par est, nisi quod tyranni in voluptatem saeviunt, reges non nisi ex causa ac necessitate? 'quid \textbf{ergo? non reges quoque occidere solent?' sed} quotiens id fieri publica utilitas persuadet; tyrannis saevitia cordi est. tyrannus autem a rege factis distat, non nomine; \textbf{nam et dionysius maior iure meritoque} praeferri multis regibus potest, et sullam tyrannum appellari quid prohibet, cui occidendi finem fecit inopia hostium? descenderit licet e dictatura sua et se togae reddiderit, quis tamen umquam tyrannus tam avide humanum sanguinem bibit quam ille, qui septem milia civium romanorum contrucidari iussit et, cum in vicino ad aedem bellonae sedens exaudisset conclamationem tot milium sub gladio gementium, exterrito senatu: 'hoc agamus' inquit, seditiosi pauculi meo iussu occiduntur'? hoc non est mentitus; pauci sullae videbantur. sed mox de sulla, cum quaeremus, quomodo hostibus irascendum sit, utique si in hostile nomen cives et ex eodem corpore abrupti transierint; interim, hoc quod dicebam, clementia efficit, ut magnum inter regem tyrannumque discrimen sit, uterque licet non minus armis valletur; sed alter arma habet, quibus in munimentum pacis utitur, alter, ut magno timore magna odia compescat, nec illas ipsas manus, quibus se commisit, securus adspicit. contrariis in contraria agitur; nam cum invisus sit, quia timetur, timeri vult, quia invisus est, et illo exsecrabili versu, qui multos praecipites dedit, utitur: oderint, dum metuant, ignarus, quanta rabies oriatur, ubi supra modum odia creverunt. temperatus enim timor cohibet animos, adsiduus vero et acer et extrema admovens in audaciam iacentes excitat et omnia experiri suadet. sic feras linea et pinnae clusas contineant, easdem a tergo eques telis incessat: temptabunt fugam per ipsa, quae fugerant, proculcabuntque formidinem. acerrima virtus est, quam ultima necessitas extundit. relinquat oportet securi aliquid metus multoque plus spei quam periculorum ostentet; alioqui, ubi quiescenti paria metuuntur, incurrere in pericula iuvat et ut aliena anima abuti. placido tranquilloque regi fida sunt auxilia sua, ut quibus ad communem salutem utatur, gloriosusque miles publicae enim securitati se dare operam videt omnem laborem libens patitur ut parentis custos; at illum acerbum et sanguinarium necesse est graventur stipatores sui. non potest habere quisquam bonae ac fidae voluntatis ministros, quibus in tormentis ut eculeo et ferramentis ad mortem paratis utitur, quibus non aliter quam bestiis homines obiectat, omnibus reis \textbf{aerumnosior ac sollicitior, ut qui} homines deosque testes facinorum ac vindices timeat, eo perductus, ut non liceat illi mutare mores. hoc enim inter cetera vel pessimum habet crudelitas: perseverandum est nec ad meliora patet regressus; scelera enim sceleribus tuenda sunt. quid autem eo \textbf{\textbf{infelicius, cui iam esse malo necesse} est?} miserabilem illum, sibi certe! nam ceteris misereri eius nefas sit, qui caedibus ac rapinis potentiam exercuit, qui suspecta sibi cuncta reddidit tam externa quam domestica, cum arma metuat, ad arma confugiens, non amicorum fidei credens, non pietati liberorum; qui, ubi circumspexit, quaeque fecit quaeque facturus est, et conscientiam suam plenam sceleribus ac tormentis adaperuit, saepe mortem timet, saepius optat, invisior \textbf{sibi quam servientibus. contrario is,} cui curae sunt universa, qui alia magis, alia minus tuetur, nullam non rei publicae partem tamquam sui nutrit, \textbf{\textbf{inclinatus ad mitiora, etiam, si} ex} usu est animadvertere, \textbf{ostendens, quam invitus aspero remedio} manus admoveat, in cuius animo nihil hostile, nihil efferum est, qui potentiam suam placide ac salutariter exercet adprobare imperia sua civibus cupiens, felix abunde sibi visus, si fortunam suam publicarit, sermone adfabilis, aditu accessuque facilis, voltu, qui maxime populos demeretur, amabilis, aequis desideriis propensus, etiam inquis non acerbus, a tota civitate amatur, defenditur, colitur. eadem de illo homines secreto loquuntur quae palam; tollere filios cupiunt et publicis malis sterilitas indicta recluditur; bene se meriturum de liberis suis quisque non dubitat, quibus tale saeculum ostenderit. hic princeps suo beneficio tutus nihil praesidiis eget, arma ornamenti causa habet. quod ergo officium eius est? quod bonorum \textbf{parentium, qui obiurgare liberos non} numquam blande, non numquam minaciter solent, aliquando admonere etiam \textbf{verberibus. numquid aliquis sanus filium a} prima offensa exheredat? nisi magnae et multae iniuriae patientiam evicerunt, nisi plus est, quod timet, quam quod damnat, non accedit ad decretorium stilum; multa ante temptat, quibus dubiam indolem et peiore iam loco positam revocet; simul deploratum est, ultima experitur. nemo ad supplicia exigenda pervenit, nisi qui remedia consumpsit. hoc, quod parenti, etiam principi faciendum est, quem appellavimus patrem patriae non adulatione \textbf{vana adducti. cetera enim cognomina honori} data sunt; magnos et felices et augustos diximus et ambitiosae \textbf{maiestati quicquid potuimus titulorum} congessimus illis hoc tribuentes; patrem quidem patriae appellavimus, ut sciret datam sibi potestatem patriam, quae est temperantissima liberis consulens suaque post illos reponens. tarde \textbf{sibi pater membra sua abscidat, etiam, cum} absciderit, reponere cupiat et in abscidendo gemat cunctatus multum diuque; prope est enim, ut libenter damnet, qui cito; prope est, ut inique puniat, qui nimis. trichonem equitem romanum memoria nostra, quia filium suum flagellis occiderat, populus graphiis in foro confodit; vix illum augusti caesaris auctoritas infestis tam patrum quam \textbf{filiorum manibus eripuit. tarium, qui} filium deprensum in parricidii \textbf{consilio damnavit causa cognita, nemo} non suspexit, quod contentus exsilio et exsilio delicato massiliae parricidam continuit et annua illi praestitit, quanta praestare integro solebat; haec liberalitas effecit, ut, in qua civitate numquam deest patronus peioribus, nemo dubitaret, quin reus merito damnatus esset, quem is pater damnare potuisset, qui odisse non poterat. hoc ipso exemplo dabo, quem compares bono patri, bonum principem. cogniturus de filio tarius advocavit in consilium caesarem augustum; venit in privatos penates, adsedit, pars alieni consilii fuit, non dixit; 'immo in meam domum veniat'; quod si factum esset, caesaris futura erat cognitio, non patris. \textbf{\textbf{audita \textbf{causa excussisque omnibus, et \textbf{his,} quae} adulescens} pro se dixerat, et his,} quibus arguebatur, petit, ut sententiam suam quisque scriberet, ne ea omnium fieret, quae caesaris fuisset; deinde, priusquam aperirentur codicilli, iuravit se tarii, hominis locupletis, hereditatem non aditurum. dicet aliquis: 'pusillo animo timuit, ne videretur locum spei suae aperire velle filii damnatione.' ego contra sentio; quilibet nostrum debuisset adversus opiniones malignas satis fiduciae habere in bona conscientia, principes multa debent etiam famae dare. iuravit se non aditurum hereditatem. tarius quidem eodem die et alterum heredem perdidit, sed caesar libertatem sententiae suae redemit; et postquam adprobavit gratuitam esse severitatem suam, quod principi semper curandum est, dixit relegandum, quo patri videretur. non culleum, non serpentes, non carcerem decrevit memor, non \textbf{de quo censeret, sed cui in \textbf{consilio esset;} mollissimo genere poenae} contentum esse debere patrem dixit in filio adulescentulo impulso in id scelus, in quo se, quod proximum erat ab innocentia, timide gessisset; debere illum ab urbe et a parentis oculis submoveri. dignum, quem in consilium patres advocarent! dignum, quem coheredem innocentibus liberis scriberent! haec clementia principem decet; quocumque venerit, mansuetiora omnia faciat. nemo regi tam vilis sit, ut illum perire non sentiat; qualiscumque pars imperii est. in magna imperia ex minoribus petamus exemplum. non unum est imperandi genus; imperat princeps civibus suis, pater liberis, praeceptor discentibus, tribunus vel centurio militibus. nonne pessimus pater videbitur, qui adsiduis plagis liberos etiam ex \textbf{levissimis causis compescet? uter autem} praeceptor liberalibus studiis dignior, qui excarnificabit discipulos, si memoria illis non constiterit aut si parum agilis in legendo oculus haeserit, an qui monitionibus et verecundia emendare ac docere malit? tribunum centurionemque da saevum: desertores faciet, quibus tamen ignoscitur. numquidnam aequum est gravius homini et durius imperari, quam imperatur animalibus mutis? atqui equum non crebris verberibus exterret domandi peritus magister; fiet enim formidolosus et contumax, nisi eum blandiente tactu permulseris. idem facit ille venator, quique instituit catulos vestigia sequi quique iam exercitatis utitur ad excitandas vel persequendas feras: nec crebro illis minatur contundet enim animos et, quicquid est indolis, comminuetur trepidatione degeneri nec licentiam vagandi errandique passim concedit. adicias his licet tardiora agentes iumenta, quae, cum ad contumeliam et miserias nata sint, nimia saevitia cogantur iugum detractare. nullum animal morosius est, \textbf{nullum maiore arte tractandum quam homo, nulli} magis parcendum. quid enim est stultius quam in iumentis quidem et canibus erubescere iras \textbf{exercere, pessima autem condicione} sub homine hominem esse? morbis medemur nec irascimur; atqui \textbf{et hic morbus est animi; mollem medicinam} desiderat ipsumque medentem minime infestum aegro. mali medici \textbf{est desperare, ne curet: idem in iis,} quorum animus adfectus est, facere debebit is, cui tradita salus omnium est, non cito spem proicere nec mortifera signa pronuntiare; luctetur cum vitiis, resistat, aliis morbum suum exprobret, quosdam molli curatione decipiat citius meliusque sanaturus remediis fallentibus; agat princeps curam non tantum salutis, sed etiam honestae cicatricis. nulla regi gloria est ex saeva animadversione quis enim dubitat posse?, at contra maxima, si vim suam continet, si multos irae alienae eripuit, neminem suae impendit. servis imperare moderate laus est. et in mancipio cogitandum est, non quantum illud impune possit pati, sed quantum tibi permittat aequi bonique natura, quae parcere etiam captivis et pretio paratis iubet. quanto iustius iubet hominibus liberis, ingenuis, honestis \textbf{non ut mancipiis abuti sed ut his, quos} gradu antecedas \textbf{quorumque tibi non servitus tradita sit, sed} tutela. servis ad statuam licet confugere; cum in servum omnia liceant, est aliquid, quod in hominem licere commune ius animantium vetet. quis non vedium pollionem peius oderat quam servi sui, quod muraenas sanguine humano saginabat et eos, qui \textbf{se aliquid \textbf{offenderant, in vivarium, quid} aliud} quam serpentium, abici iubebat? hominem mille mortibus dignum, sive devorandos servos obiciebat muraenis, quas esurus erat, sive in hoc tantum illas alebat, ut sic aleret. quemadmodum domini crudeles tota civitate commonstrantur invisique et detestabiles sunt, ita regum et iniuria latius patet et infamia atque odium saeculis traditur; quanto autem non nasci melius \textbf{fuit, \textbf{quam numerari inter publico malo} natos!} excogitare nemo quicquam poterit, quod magis decorum regenti sit quam clementia, quocumque modo is et quocumque iure praepositus ceteris erit. eo scilicet formosius id esse magnificentiusque fatebimur, quo in maiore praestabitur potestate, quam non oportet noxiam esse, si ad naturae legem componitur. natura enim commenta est regem, quod et ex aliis animalibus licet cognoscere et ex apibus; quarum regi amplissimum cubile est medioque ac tutissimo loco; praeterea opere vacat exactor alienorum operum, et amisso rege totum dilabitur, nec umquam plus unum patiuntur melioremque pugna quaerunt; praeterea insignis regi forma est dissimilisque ceteris cum magnitudine tum nitore. hoc tamen maxime distinguitur: iracundissimae ac pro corporis captu pugnacissimae sunt apes et aculeos in volnere relinquunt, rex ipse sine aculeo est; noluit illum natura nec saevum esse nec ultionem magno constaturam petere telumque detraxit et iram eius inermem reliquit. exemplar hoc magnis regibus ingens; est enim illi mos exercere se in \textbf{\textbf{parvis et ingentium rerum documenta} in} minima parere. pudeat ab exiguis animalibus non trahere mores, cum tanto hominum moderatior esse animus debeat, quanto vehementius nocet. utinam quidem eadem homini lex esset et ira cum telo suo frangeretur nec saepius liceret nocere quam semel nec alienis viribus exercere odia! facile enim lassaretur furor, si per se sibi satis faceret et si mortis periculo vim suam effunderet. sed ne nunc quidem illi cursus tutus est; tantum enim necesse est timeat, quantum timeri voluit, et manus \textbf{omnium observet et eo quoque tempore, quo non} captatur, peti se iudicet nullumque momentum immune a metu habeat. hanc aliquis agere vitam sustinet, cum liceat innoxium aliis, ob hoc securum, salutare potentiae ius laetis omnibus tractare? errat enim, si quis existimat tutum esse ibi regem, ubi nihil a rege tutum est; securitas securitate mutua paciscenda est. non opus est instruere in altum editas arces nec in adscensum arduos colles emunire \textbf{nec latera montium abscidere, multiplicibus} se muris turribusque saepire: salvum regem clementia in aperto praestabit. unum est inexpugnabile munimentum amor civium. quid pulchrius est quam vivere optantibus cunctis et vota non sub custode nuncupantibus? si paulum valetudo titubavit, non spem hominum excitari, sed metum? nihil esse cuiquam tam pretiosum, quod non pro salute praesidis sui commutatum velit? ne ille, cui contingit ut sibi quoque \textbf{vivere debeat? in hoc adsiduis bonitatis} argumentis probavit non rem publicam suam esse, sed se rei publicae. quis huic audeat struere aliquod periculum? quis ab hoc non, si \textbf{possit, fortunam quoque avertere velit, sub} quo iustitia, pax, pudicitia, securitas, dignitas florent, sub quo opulenta civitas copia bonorum omnium abundat? nec alio animo rectorem suum intuetur, quam si di immortales potestatem visendi sui faciant, intueamur venerantes colentesque. quid autem? non proximum illis locum tenet is, qui se ex deorum natura gerit, beneficus ac largus et in melius potens? hoc adfectare, hoc imitari decet, maximum ita haberi, ut optimus simul habeare. duabus causis punire princeps solet, si \textbf{aut se vindicat aut alium. prius de ea parte} \textbf{disseram, quae ipsum contingit; difficilius} est enim moderari, ubi dolori debetur ultio, quam ubi exemplo. supervacuum est hoc loco admonere, ne facile credat, ut verum excutiat, ut innocentiae faveat et, ut appareat, non minorem agi rem periclitantis quam iudicis sciat; hoc enim ad iustitiam, non ad clementiam pertinet; nunc illum hortamur, ut manifeste laesus animum in potestate habeat et poenam, si tuto poterit, donet, si minus, temperet longeque sit in suis quam in alienis iniuriis exorabilior. nam quemadmodum non est magni animi, qui de alieno liberalis est, sed ille, qui, quod alteri donat, sibi detrahit, ita clementem vocabo non in alieno dolore facilem, sed eum, qui, cum suis stimulis exagitetur, non prosilit, qui intellegit magni animi esse iniurias in summa potentia pati nec quicquam esse gloriosius principe impune laeso. ultio duas praestare res solet: aut solacium \textbf{adfert ei, qui accepit iniuriam, aut in} reliquum securitatem. principis maior est fortuna, quam ut solacio egeat, manifestiorque vis, quam ut alieno malo opinionem sibi virium quaerat. hoc dico, cum ab inferioribus petitus violatusque est; nam si, quos pares aliquando habuit, infra se videt, satis vindicatus est. regem et servus occidit et serpens et sagitta; servavit quidem nemo nisi maior eo, quem servabat. uti itaque animose debet tanto munere deorum dandi auferendique vitam potens. in \textbf{iis praesertim, quos scit aliquando sibi par} fastigium obtinuisse, hoc arbitrium adeptus ultionem implevit perfecitque, quantum verae poenae satis erat; perdidit enim vitam, \textbf{qui debet, et, quisquis ex alto \textbf{ad inimici} pedes abiectus alienam} de capite regnoque sententiam exspectavit, in servatoris sui gloriam vivit plusque eius nomini confert incolumis, quam \textbf{si \textbf{ex oculis ablatus esset. adsiduum enim}} spectaculum alienae virtutis est; in triumpho cito transisset. si vero regnum quoque suum tuto relinqui apud eum potuit reponique eo, unde deciderat, ingenti incremento surgit \textbf{laus eius, qui contentus fuit ex rege victo} nihil praeter gloriam sumere. hoc est etiam ex victoria sua triumphare testarique nihil se, quod dignum esset victore, apud victos invenisse. cum civibus et ignotis atque humilibus eo moderatius agendum est, quo minoris est adflixisse eos. quibusdam libenter parcas, a quibusdam te vindicare fastidias et non aliter quam ab animalibus parvis sed obterentem inquinantibus reducenda manus est; at in iis, qui in ore civitatis servati punitique erunt, occasione notae clementiae utendum est. transeamus ad alienas iniurias, in quibus vindicandis haec tria lex secuta est, quae princeps quoque sequi debet: aut ut eum, quem punit, emendet, \textbf{aut ut poena eius ceteros meliores} reddat, aut ut sublatis malis securiores ceteri vivant. ipsos facilius emendabis minore poena; diligentius enim vivit, cui aliquid integri superest. nemo dignitati perditae parcit; impunitatis genus est iam non habere poenae locum. civitatis autem mores magis corrigit parcitas animadversionum; facit enim consuetudinem peccandi multitudo peccantium, et minus gravis nota est, quam turba damnationum levat, et severitas, quod maximum remedium habet, adsiduitate amittit auctoritatem. constituit bonos mores civitati princeps et vitia eluit, si patiens eorum est, non tamquam probet, sed tamquam invitus et cum magno tormento ad castigandum veniat. verecundiam \textbf{peccandi facit ipsa clementia regentis;} gravior multo poena videtur, quae a miti viro constituitur. praeterea videbis ea saepe committi, quae saepe vindicantur. pater tuus plures intra quinquennium culleo insuit, quam omnibus saeculis insutos accepimus. multo minus audebant liberi nefas ultimum admittere, quam diu sine lege crimen fuit. \textbf{summa enim prudentia altissimi viri et} rerum naturae peritissimi maluerunt velut incredibile scelus et ultra audaciam positum praeterire quam, dum vindicant, ostendere posse fieri; itaque parricidae cum lege coeperunt, et illis facinus poena monstravit; pessimo vero loco pietas fuit, postquam saepius culleos vidimus quam cruces. in qua civitate raro homines puniuntur, in ea consensus fit innocentiae et indulgetur velut publico bono. putet se innocentem esse civitas, erit; magis irascetur a communi frugalitate desciscentibus, si paucos esse eos viderit. periculosum est, mihi crede, ostendere civitati, \textbf{\textbf{quanto plures mali sint. dicta est aliquando} a} senatu sententia, ut servos a liberis cultus distingueret; deinde apparuit, quantum periculum immineret, si servi nostri numerare nos coepissent. idem scito metuendum esse, si nulli ignoscitur; cito apparebit, pars civitatis deterior quanto praegravet. non minus principi turpia sunt multa supplicia quam medico multa funera; remissius imperanti melius paretur. natura contumax est humanus animus et in \textbf{contrarium atque arduum nitens sequiturque} facilius quam ducitur; et ut generosi ac nobiles equi melius facili freno reguntur, ita clementiam voluntaria innocentia impetu suo sequitur, et dignam putat civitas, quam servet sibi. plus itaque hac via proficitur. crudelitas minime humanum malum est indignumque tam miti animo; ferina ista rabies est sanguine gaudere ac volneribus et abiecto homine \textbf{in silvestre animal transire. quid enim interest,} oro te, alexander, leoni lysimachum obicias an ipse laceres dentibus tuis? tuum illud os est, tua illa feritas. quam cuperes tibi potius ungues esse, tibi rictum illum edendorum hominum capacem! non exigimus a te, ut manus ista, exitium familiarium certissimum, ulli salutaris sit, ut iste animus ferox, insatiabile gentium malum, citra sanguinem caedemque satietur; clementia iam vocatur, ad occidendum amicum cum carnifex inter homines eligitur. hoc est, quare vel maxime abominanda sit saevitia, quod excedit fines primum solitos, deinde humanos, nova supplicia conquirit, ingenium advocat, ut instrumenta excogitet, per quae varietur atque extendatur dolor, delectatur malis hominum; tunc illi dirus animi morbus ad insaniam pervenit ultimam, cum crudelitas versa est in voluptatem et iam occidere hominem iuvat. matura talem virum a tergo sequitur aversio, odia, venena, gladii; tam multis periculis petitur, quam multorum ipse periculum est, privatisque non numquam consiliis, alias vero consternatione publica circumvenitur. levis enim et privata pernicies non totas urbes movet; quod late furere coepit et omnes adpetit, undique configitur. serpentes parvolae fallunt nec publice conquiruntur; ubi aliqua solitam mensuram \textbf{transit et in \textbf{monstrum \textbf{excrevit, ubi fontes} sputu inficit et, si}} adflavit, deurit obteritque, quacumque incessit, ballistis petitur. possunt verba dare et evadere pusilla mala, ingentibus obviam itur. sic unus aeger ne domum quidem perturbat; at ubi crebris mortibus pestilentiam esse apparuit, conclamatio civitatis ac fuga est, et dis ipsis manus intentantur. sub uno aliquo tecto flamma apparuit: familia vicinique aquam ingerunt; at incendium vastum et multas iam domos depastum parte urbis obruitur. crudelitatem privatorum quoque serviles manus sub certo crucis periculo ultae sunt; tyrannorum gentes populique et, quorum erat malum, et ei, quibus imminebat, exscindere adgressi sunt. aliquando sua praesidia \textbf{\textbf{in ipsos consurrexerunt perfidiamque} et} impietatem \textbf{et feritatem et, quidquid ab illis didicerant,} in ipsos exercuerunt. quid enim \textbf{potest quisquam ab \textbf{eo sperare, quem malum esse} docuit? non} diu nequitia apparet nec, quantum iubetur, peccat. sed puta esse tutam crudelitatem, quale eius regnum est? non aliud quam captarum urbium forma et terribiles facies publici metus. omnia maesta, trepida, confusa; voluptates ipsae timentur; non convivia securi ineunt, in quibus lingua \textbf{\textbf{sollicite etiam ebriis custodienda} est,} non spectacula, ex quibus materia criminis ac periculi quaeritur. apparentur licet magna impensa et regiis opibus et artificum exquisitis nominibus, quem tamen ludi in carcere iuvent? quod istud, di boni, malum est occidere, saevire, delectari sono catenarum et civium capita decidere, quocumque ventum est, multum sanguinis fundere, adspectu suo terrere ac fugare! quae alia vita esset, si leones ursique regnarent, si serpentibus in nos ac noxiosissimo cuique animali daretur potestas? illa rationis expertia et a nobis immanitatis crimine \textbf{damnata \textbf{abstinent suis, et tuta est etiam inter}} feras similitudo: horum ne a necessariis quidem sibi rabies temperat, sed externa suaque in aequo habet, quo plus se exercitat, eo incitatior. singulorum deinde caedibus in exitia gentium serpit et inicere tectis ignem, aratrum vetustis urbibus inducere potentiam putat; et unum occidi iubere aut alterum parum imperatorium credit; nisi eodem tempore grex miserorum sub ictu stetit, crudelitatem suam in ordinem coactam putat. felicitas illa multis salutem dare et ad vitam ab ipsa morte revocare et mereri clementia civicam. nullum ornamentum principis fastigio dignius pulchriusque est quam illa corona \textbf{ob cives servatos, non hostilia arma} detracta victis, non currus barbarorum sanguine cruenti, non parta bello spolia. haec divina potentia est gregatim ac publice servare; multos quidem occidere et indiscretos incendii ac ruinae potentia est. \textbf{ut de clementia scriberem, nero caesar,} una me vox tua maxime compulit, quam ego non sine admiratione \textbf{et, cum diceretur, audisse memini \textbf{et} \textbf{deinde aliis narrasse, vocem generosam,}} magni animi, magnae lenitatis, quae non composita nec alienis auribus data subito erupit et bonitatem tuam cum fortuna tua litigantem in medium adduxit. animadversurus in latrones duos burrus praefectus tuus, vir egregius et tibi principi natus, exigebat a te, scriberes, in quos et ex qua \textbf{\textbf{causa animadverti velles; hoc saepe dilatum} ut} aliquando fieret, instabat. invitus invito cum chartam protulisset traderetque, exclamasti: 'vellem litteras nescirem!' dignam vocem, quam audirent omnes gentes, quae romanum imperium incolunt quaeque iuxta iacent dubiae libertatis quaeque se contra viribus aut animis attollunt! vocem in contionem omnium mortalium mittendam, in cuius verba principes regesque iurarent! vocem publica generis humani innocentia dignam, cui redderetur antiquum illud saeculum! \textbf{nunc profecto consentire decebat ad aequum} bonumque expulsa alieni cupidine, ex qua omne animi malum oritur, pietatem integritatemque cum fide ac modestia resurgere et vitia diuturno abusa regno dare tandem felici ac puro saeculo locum. futurum hoc, caesar, ex \textbf{magna parte sperare et confidere libet.} tradetur ista animi tui mansuetudo diffundeturque paulatim per omne imperii corpus, et cuncta in similitudinem tuam formabuntur. capite bona valetudo: inde omnia vegeta sunt atque erecta aut languore demissa, prout animus eorum vivit aut marcet. erunt cives, erunt socii digni hac bonitate, et in totum orbem recti mores revertentur; parcetur ubique manibus tuis. diutius me morari hic patere, non ut blandum auribus tuis nec enim hic mihi mos est; maluerim veris offendere quam placere adulando; quid ergo est? praeter id, quod bene factis dictisque tuis quam familiarissimum esse te cupio, ut, quod nunc natura et impetus est, fiat \textbf{iudicium, illud mecum considero multas} voces magnas, sed detestabiles, \textbf{in \textbf{vitam humanam pervenisse celebresque}} volgo \textbf{ferri, ut illam: 'oderint, \textbf{dum metuant,' cui graecus} versus similis est,} qui se mortuo terram misceri ignibus iubet, et alia huius notae. ac nescio quomodo ingenia in immani et invisa materia secundiore ore expresserunt sensus vehementes et concitatos; nullam adhuc vocem audii ex bono lenique animosam. quid ergo est? ut raro, invitus et cum magna cunctatione, ita aliquando scribas necesse est istud, quod tibi in odium litteras adduxit, sed, sicut facis, cum magna cunctatione, cum multis dilationibus. et ne forte decipiat nos speciosum clementiae nomen aliquando et in contrarium abducat, videamus, quid sit clementia qualisque sit et quos fines habeat. clementia est temperantia animi in potestate ulciscendi vel lenitas superioris adversus inferiorem in constituendis poenis. plura proponere tutius est, ne una finitio parum rem comprehendat et, ut ita dicam, formula excidat; itaque dici potest et inclinatio animi ad lenitatem in poena exigenda. illa finitio contradictiones inveniet, quamvis maxime ad verum accedat, si dixerimus clementiam esse moderationem aliquid ex merita ac debita poena remittentem: reclamabitur nullam virtutem cuiquam minus debito facere. atqui hoc omnes intellegunt clementiam esse, quae se flectit citra id, quod merito constitui posset. huic contrariam imperiti putant severitatem; \textbf{sed \textbf{nulla virtus virtuti contraria est.} quid} ergo opponitur clementiae? crudelitas, quae nihil aliud est quam atrocitas animi in exigendis poenis. 'sed quidam non exigunt poenas, crudeles tamen sunt, tamquam qui ignotos homines et obvios non in compendium, sed occidendi causa occidunt nec interficere contenti saeviunt, ut busiris ille et procrustes et piratae, qui captos verberant et in ignem vivos imponunt.' haec crudelitas quidem; sed quia nec ultionem sequitur non enim laesa est nec peccato alicui irascitur nullum enim antecessit crimen, extra finitionem nostram cadit; finitio enim continebat in poenis exigendis intemperantiam animi. possumus dicere non esse hanc crudelitatem, sed feritatem, cui voluptati saevitia est; possumus insaniam vocare: nam varia sunt genera eius et nullum certius, \textbf{quam quod in caedes hominum \textbf{et lancinationes} pervenit. illos} ergo crudeles vocabo, qui puniendi causam habent, modum non habent, sicut in phalari, quem aiunt non quidem in homines innocentes, sed super humanum ac probabilem modum saevisse. possumus effugere cavillationem et ita finire, ut sit crudelitas inclinatio animi ad asperiora. hanc clementia repellit longe iussam stare a se; cum severitate illi convenit. ad rem pertinet quaerere hoc loco, quid sit misericordia; plerique enim ut virtutem eam laudant et bonum hominem vocant misericordem. et haec vitium animi est. utraque circa severitatem circaque clementiam posita \textbf{sunt, quae vitare debemus; per speciem enim} severitatis in crudelitatem incidimus, per speciem clementiae in misericordiam. in hoc leviore periculo erratur, sed par error est a vero recedentium. ergo quemadmodum religio deos colit, superstitio violat, ita clementiam mansuetudinemque omnes boni viri praestabunt, misericordiam autem vitabunt; est enim vitium pusilli animi ad speciem alienorum malorum succidentis. itaque pessimo cuique familiarissima est; anus et mulierculae sunt, quae lacrimis nocentissimorum moventur, quae, si liceret, carcerem effringerent. misericordia non causam, sed \textbf{fortunam spectat; clementia rationi} accedit. scio male audire apud imperitos sectam stoicorum tamquam duram nimis et minime principibus regibusque bonum daturam consilium; obicitur illi, quod sapientem negat misereri, negat ignoscere. haec, si per se ponantur, invisa sunt; videntur enim nullam relinquere spem humanis erroribus, sed \textbf{\textbf{omnia delicta ad poenam deducere. quod si} est} quidnam haec scientia, quae dediscere humanitatem iubet portumque adversus fortunam certissimum mutuo auxilio cludit? sed nulla secta benignior leniorque est, nulla amantior hominum et communis boni attentior, ut propositum sit usui esse et auxilio nec sibi tantum, sed universis singulisque consulere. misericordia est aegritudo animi ob alienarum miseriarum speciem aut tristitia ex alienis malis contracta, quae accidere immerentibus credit; aegritudo autem in sapientem virum non cadit; serena \textbf{eius mens est, nec quicquam incidere potest, quod} illam obducat. nihilque aeque hominem quam magnus animus decet; non potest autem magnus esse idem ac maestus. maeror contundit mentes, abicit, contrahit; hoc sapienti ne in suis quidem accidet calamitatibus, sed omnem fortunae iram reverberabit et ante se franget; eandem semper faciem \textbf{servabit, placidam, inconcussam, quod facere \textbf{non} posset, si \textbf{tristitiam reciperet. adice,} quod sapiens et} providet et in expedito consilium habet; numquam autem liquidum sincerumque ex turbido venit. tristitia inhabilis est ad dispiciendas res, utilia excogitanda, periculosa vitanda, aequa aestimanda; ergo non miseretur, quia id sine miseria animi non fit. cetera omnia, quae, qui miserentur, volo facere, libens et altus animo faciet; succurret alienis lacrimis, non accedet; dabit manum naufrago, exsuli hospitium, egenti stipem, non hanc contumeliosam, quam \textbf{pars \textbf{maior horum, qui misericordes videri}} volunt, abicit et fastidit, quos adiuvat, contingique ab iis timet, sed ut homo homini ex communi dabit; donabit lacrimis maternis filium et catenas solvi iubebit et ludo eximet et cadaver etiam noxium sepeliet, sed faciet \textbf{ista tranquilla mente, voltu suo. ergo non} miserebitur sapiens, sed succurret, sed proderit, in commune auxilium natus ac bonum publicum, ex quo dabit cuique partem. etiam ad calamitosos pro portione \textbf{improbandosque et emendandos bonitatem} suam permittet; adflictis vero et forte laborantibus multo libentius subveniet. quotiens poterit, fortunae intercedet; ubi enim opibus potius utetur aut viribus, quam ad restituenda, quae casus impulit? voltum quidem non deiciet nec animum ob crus alicuius aridum aut pannosam maciem et innixam baculo senectutem; ceterum omnibus dignis proderit et deorum more calamitosos propitius respiciet. misericordia vicina est miseriae; habet enim aliquid trahitque ex ea. imbecillos oculos esse scias, qui ad alienam lippitudinem et ipsi subfunduntur, tam mehercules quam morbum esse, non hilaritatem, semper adridere ridentibus et ad omnium oscitationem ipsum quoque os diducere; misericordia vitium est animorum nimis miseria paventium, quam si quis a sapiente exigit, prope est, ut lamentationem exigat et in alienis funeribus gemitus. 'at \textbf{quare non ignoscet?' agedum constituamus} nunc quoque, quid sit venia, et sciemus dari illam a sapiente non debere. venia est poenae meritae remissio. \textbf{hanc sapiens quare non debeat dare, reddunt} rationem diutius, quibus hoc \textbf{propositum est; ego ut breviter tamquam in alieno} iudicio dicam: ei ignoscitur, qui puniri debuit; sapiens autem nihil facit, quod non debet, nihil praetermittit, quod debet; itaque poenam, quam exigere debet, non donat. sed illud, quod ex venia consequi vis, honestiore tibi via tribuet; parcet enim sapiens, consulet et corriget; idem faciet, quod, si ignosceret, nec ignoscet, quoniam, qui ignoscit, fatetur aliquid se, quod fieri debuit, omisisse. aliquem verbis tantum admonebit, poena non adficiet aetatem eius emendabilem intuens; aliquem invidia criminis manifeste laborantem iubebit incolumem esse, quia deceptus est, quia per vinum \textbf{lapsus; hostes dimittet salvos, aliquando} etiam laudatos, si honestis causis pro fide, pro foedere, pro libertate in \textbf{bellum \textbf{acciti sunt. haec omnia non veniae, sed}} clementiae opera sunt. clementia liberum arbitrium habet; non sub formula, sed ex aequo et bono iudicat; et absolvere illi licet \textbf{et, quanti vult, taxare litem. nihil ex his} facit, tamquam iusto minus fecerit, sed tamquam id, quod constituit, iustissimum sit. ignoscere autem est, quem iudices puniendum, non punire; venia debitae poenae remissio est. clementia hoc primum praestat, ut, quos dimittit, nihil aliud illos pati debuisse pronuntiet; plenior est quam venia, honestior est. de verbo, ut mea fert opinio, controversia est, de re quidem convenit. sapiens multa remittet, multos parum sani, sed sanabilis ingenii servabit. agricolas bonos imitabitur, qui non tantum rectas procerasque arbores colunt; illis quoque, quas aliqua depravavit causa, adminicula, quibus derigantur, adplicant; alias circumcidunt, ne proceritatem rami premant, quasdam infirmas vitio loci nutriunt, quibusdam aliena umbra laborantibus caelum aperiunt. videbit, quod ingenium qua ratione tractandum sit, quo modo in rectum prava flectantur.
