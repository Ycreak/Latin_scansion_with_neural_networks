\section{Introduction}
Automated scansion of poetry by computer is a long-standing goal, and a difficult task. Antje Wessels and Matthew Payne (Classics) and Luuk Nolden and Philippe Bors (Computer Science), who have developed a digital platform specifically for Classical fragments (oscc.lucdh.nl), now plan the implementation of a digital tool to automatically scan Latin poetry. They plan to utilize machine learning to bring a more sophisticated approach which plays to the strengths of artificial intelligence.\\

\textbf{Why is the study of meter important?}
The sound of a text is an essential part of its content. Similar to what we experience when we listen to different musical styles, our perception of a text depends on the rhythmical patterns in which this text is presented. Different rhythms lead to different expectations about its content, and across literatures and cultures specific rhythms have always been associated with specific genres. This was as true in the ancient world as in the modern. Just as we expect something very different from hearing the rhythms of slam poetry than those of a limerick or from those of a ballad, so an ancient audience would have instantly recognized a line of the epic writer Ennius from a speech of a slave in Plautine comedy - and the Roman writer Cicero tells us that his fellow Romans were sensitive enough to the meanings of rhythms that they could instantly sense when a writer was subverting expectations through his use of metrical patterns.\\

\textbf{Why Latin poetry?}
Latin poetry features a variety of authors and works which use a wide range of different poetic meters. Actually, rhythm was everywhere in Roman literature, even in prose it was employed as a crucial element: As masters of rhetoric such as Cicero stressed, rhythm, though it should be used subtly, is one of the qualities of good speeches. In Latin poetry different poetic meters - different arrangements of rhythmic patterns - marked out different genres, although the principles of how the sounds of poetry created rhythms remained the same. So Latin poetry provides a large amount of different material written in different poetic meters. Moreover, some of these meters featured simpler rhythms with less metrical variation, others far more complex rhythms with lots of variation. This makes them an excellent set of data for developing a complex scansion tool.\\

\textbf{What tools from the digital humanities will you use?}
Because Latin poetry contains lots of different meters, it is more important to understand the principles by which the sounds of words create the rhythms than any specific patterns that may only work for one author or work. Just like a student needs to learn to apply these principles and to get a feeling for the relationship between these principles and how they work in practice in texts, so too for automatic scansion to succeed, the machine needs to be able to translate between these fundamental principles and specific metrical patterns as they are found in different authors and genres. This is why we will utilize approaches from machine learning, rather than trying to create a model that follows rules which only work for one kind of metrical pattern. Our current plan is to create a model which can train itself using techniques such as Markov chains and SVM, in order that it be able to handle metrical ambiguities and exceptions. In this way the model should even be able to handle a completely new kind of meter.\\

\textbf{How will you go about this?}
Just like we teach a student first how to scan more straightforward metrical lines like the dactylic hexameter before going on to rather complex metrical forms such as found in the odes of Horace or in the polymetrical cantica of Plautus, we are going to train our machine on the corpora of Virgil (written in dactylic hexameter), then introduce a different meter, iambic trimeter, found in the tragedies of Seneca, before finally training the model on the more complex metrical forms of Plautus.\\

\textbf{What are some interesting applications?}
Not all Latin poetry survives to us as complete texts - many lines come to us in the form of fragments, because we find them as quotations in prose texts. Fragments are typically preserved in incomplete metrical units, such as half-lines. They are also more vulnerable to textual corruption, because they were quite likely to be misunderstood or confused with the prose text they were quoted within. One exciting possibility is that the model will help us to better identify the metrical pattern of a corrupt line.\\

Even more exciting is the possibility of using the model to detect new fragments. There are many prose texts which scholars believe may embed verse lines, but it is difficult and time-consuming for an individual to look through all of them. Applying the model to these texts could flag passages which seem to embed such lines.\\

\textbf{What relevance does this have beyond Latin poetry?}
A model trained to successfully recognize and extrapolate rhythmic information from texts written in Latin could likely be adapted to work in any language. The adaptation might involve changing the model from analyzing quantity (as in Latin) to analyzing stress (as, for instance, in most Dutch or English poetry), but, while a rules-based approach would require starting from scratch, the machine learning approach could accommodate this.\\

\textbf{Who will benefit from this project?}
Any student of Latin poetry, at whatever level, who is trying to get a better understanding of metre and its relevance to the content of poetry will gain from applying the model to the texts they are learning about. Rather than seeing scansion as a mechanistic process of applying rules, by experimenting with the scansion tool they will hopefully gain greater insight into the possibilities and complexities of different metres.\\

As we said at the beginning, the sound of a text is an essential part of its content. By applying machine learning techniques to this problem of metre through the large and multi-faceted body of data provided by Latin poetry we aim to test their usefulness and extensibility. And we hope that any researchers in the humanities who are working on topics where sound is connected with meaning will benefit from our exploration of developing a model through machine learning.\\